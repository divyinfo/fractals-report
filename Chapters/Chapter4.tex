% Chapter Template

\chapter{Implementation} % Main chapter title

\label{Chapter4} % Change X to a consecutive number; for referencing this chapter elsewhere, use \ref{ChapterX}

In this chapter, we'll describe in details the overall structure of the project, how the files are arranged and the functionalities of each components.

%----------------------------------------------------------------------------------------
%	SECTION 1
%----------------------------------------------------------------------------------------

\section{Front End}

Since this project is a pure web project, the front end occupies a large portion of the codes.

%-----------------------------------
%	SUBSECTION 1
%-----------------------------------
\subsection{HTML Entry \texttt{index.html}}

The entry of the project is where this program gets started, in similar concept of the \texttt{main()} function in \texttt{C} or the \texttt{public static void main(String[] args)} function in \texttt{Java}. The entry point is a \gls{html} file and as expected named \texttt{index.html}. It introduces the front end structure of the project in raw.

%-----------------------------------
%	SUBSECTION 2
%-----------------------------------

\subsection{Main JavaScript \texttt{index.js}}

At the very end of \texttt{index.html}, the main \gls{js} file \texttt{index.js} gets included via a \texttt{<script>} tag.

// Reasons to put at the end goes here

%----------------------------------------------------------------------------------------
%	SECTION 2
%----------------------------------------------------------------------------------------

\section{Back End Calculation}

The back end calculation is done in the \gls{js} file \texttt{naive-worker.js}. This file is being used for initializing the \emph{WebWorker}s inside \texttt{index.js} dynamically. Whenever a calculation or extraction for a specific region of a dataset is needed, the main \gls{js} file \texttt{index.js} is going to send a message to \texttt{naive-worker.js} with desired parameters and this back end will respond with corresponding image data.

\subsection{Global Scope}

In the global scope of this file, the following things were done.

\paragraph{Includes} The \texttt{decimal.js} dependency is included for high-precision floating points calculation. Default parameters for the dependency is set.

\paragraph{Constants} Constants of default screen width and default screen height are defined in case the front end doesn't give these parameters.

\paragraph{Canvas} An \texttt{OffscreenCanvas} instance is created and instantiated with the dimensions of by default the values of the defined constants. The \texttt{OffscreenCanvas} will be used as the canvas to generate the desired image on, and since it's not being shown on the screen, will occupy less system resources and boost the calculation speed. Corresponding variables will be declared, respectively \texttt{canvas} for the \texttt{OffscreenCanvas} itself and \texttt{ctx} as the 2d context of the canvas.

\subsection{Message Reception}

\subsection{Iteration Limit}

\subsection{Iteration Count for One Point}

\subsection{Image Generation}

\subsection{High Precision Version}

\section{Utility Assets}

Other open source third-party utilities lie in different folders with corresponding names.

\subsection{Folder \texttt{./js}}

In \texttt{./js} folder, all \gls{js} third-party files are here, including:

\begin{itemize}
    \item File \texttt{decimal.min.js} is for high-precision floating points calculation for \glsdesc{js}.
    \item File \texttt{jquery-3.4.1.min.js} is for \gls{dom} traversal and manipulation, event handling and animation.
    \item File \texttt{bootstrap.bundle.min.js} is for some basic styling of the control panel sitting on top right corner of the screen.
\end{itemize}

\subsection{Folder \texttt{./fa}}

\subsection{Folder \texttt{./css}}